 \documentclass[12pt]{article}
\usepackage[T2A]{fontenc}
\usepackage[utf8]{inputenc}       

\usepackage[english]{babel}
\usepackage{amsmath,amsfonts,amsthm,amssymb,amsbsy,amstext,amscd,amsxtra,multicol}
\usepackage{verbatim}
\usepackage{tikz}
\usetikzlibrary{automata,positioning}
\usepackage{multicol}
\usepackage{graphicx}
\usepackage[colorlinks,urlcolor=blue]{hyperref}
\usepackage[stable]{footmisc}
\usepackage{ dsfont }
\usepackage{wrapfig}
\usepackage{xparse}
\usepackage{ifthen}
\usepackage{bm}
\usepackage{color}

\usepackage{algorithm}
\usepackage{algpseudocode}
	
\usepackage{xcolor}
\usepackage{hyperref}
\definecolor{linkcolor}{HTML}{799B03} % цвет гиперссылок
\definecolor{urlcolor}{HTML}{799B03} % цвет гиперссылок
 
%\hypersetup{pdfstartview=FitH,  linkcolor=linkcolor,urlcolor=urlcolor, colorlinks=true}

\newtheorem{theorem}{Theorem}[section]
\newtheorem{lemma}{Lemma}[section]

\DeclareMathOperator{\sign}{sign}
\DeclareMathOperator{\grad}{grad}
\DeclareMathOperator{\intt}{int}
\DeclareMathOperator{\conv}{conv}
\begin{document}

\section{О вычислении функции}
Пусть $f$ - вычисляемая функция, $\tilde f$ - ее приближение. Определим следующие функии:

$$\epsilon(x) = \Big|f(x)-\tilde{f}(x)\Big|\text{-- ошибка приближения}$$
$$\delta_f(x_1,x_2) = f(x_1)-f(x_2)$$

Для GSS, дихотомии и им подобных методов одномерной оптимизации интересен знак $\delta_f$. Поэтому когда мы используем ее приближение, нам достаточно:

$$\sign\left(\delta_f(x_1,x_2)\right) = \sign\left(\delta_{\tilde{f}}(x_1,x_2)\right)$$

Более сильное условие:

$$\Big|\delta_f(x_1,x_2) - \delta_{\tilde{f}}(x_1,x_2)\Big|\leq \Big|\delta_{f}(x_1,x_2)\Big|$$

Но как его достичь? 

\begin{gather}
\Big|\delta_f(x_1,x_2) - \delta_{\tilde{f}}(x_1,x_2)\Big| \leq\\
\epsilon(x_1)+\epsilon(x_2)
\end{gather}

Таким образом получаем следующее условие на точность вычисления функции:

$$\epsilon(x_1)+\epsilon(x_2)\leq \Big|\delta_{f}(x_1,x_2)\Big|$$

Если $f$ -- $L$-липшецева функция, то получаем удобное неравенство:

$$\boxed{\epsilon(x_1)+\epsilon(x_2)\leq L\Big|x_1-x_2\Big|}$$

\section{О вычислении производных}

\subsection{В решении на отрезке}

Лагранжиан прямой задачи:

$$\Phi(\textbf{x}, \lambda) = f(\textbf{x}) + \lambda_1g_1(\textbf{x})+\lambda_2g_2(\textbf{x})$$

Пусть нас интересует производная по $\lambda_2$ в точке $\lambda$ функции $\phi(\lambda) = \min \Phi(\textbf{x}, \lambda)$ (для упрощения записи $\phi'$). Мы хотим вычислить её приближение $\tilde{\phi'}$ так, чтобы знак совпадал. Для этого достаточно:

$$|\phi' - \tilde{\phi}'|\leq |\tilde{\phi}'|$$

$$|g_2(\textbf{x}^*)-g_2(\textbf{x})|\leq |\tilde{\phi}'|$$

Пользуясь липшецевостью $g_2$ получаем достаточное условие:

$$\boxed{L_{g_k}\|\textbf{x}^*-\textbf{x}\|\leq |g_k(\textbf{x})|}$$

Аналогично для производной по $\lambda_1$.

\subsection{В current point on segment}

У нас есть условие на точность решения на отрезке:

$$\|\lambda_1-\lambda_2\|=\delta = \leq \frac{\phi'(\lambda_{cur})}{L}=\epsilon,$$

где $\lambda_cur\in[\lambda_1, \lambda_2]$ -текущее приближение решения, производная берется по соответствующей координате в зависимости от отрезка. Пока что выбирается как центр отрезка.

По другому это условие выглядит так:

$$\delta-\epsilon<0$$

Мы можем вычислить $\epsilon$ приблизительно и получить $\tilde{\epsilon}$. По этому приближению мы должны проверить выше написанное условие, т.е. знаки выражений $\delta-\epsilon$ и $\delta - \tilde{\epsilon}$ должны совпадать. Для этого достаточно:

$$\Big|(\delta-\epsilon)-(\delta - \tilde{\epsilon})\Big|\leq |\delta - \tilde{\epsilon}|$$

$$|\epsilon-\tilde{\epsilon}|\leq |\delta - \tilde{\epsilon}|$$

С учетом того, что для двойственных задач $\phi_{\lambda_k}'(\lambda)=g_k\left(\textbf{x}(\lambda)\right)$ и если $g_k$ $L_{g_k}$-липшецева получаем следующее достаточное условие:

$$\boxed{L_{g_k}\|\textbf{x}^*-\textbf{x}\|}\leq \Big||\lambda_1-\lambda_2| - \frac{g_k(\textbf{x})}{L}\Big|$$

\end{document}
