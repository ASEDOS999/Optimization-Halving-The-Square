 \documentclass[12pt]{article}
\usepackage[T2A]{fontenc}
\usepackage[utf8]{inputenc}       

\usepackage[english]{babel}
\usepackage{amsmath,amsfonts,amsthm,amssymb,amsbsy,amstext,amscd,amsxtra,multicol}
\usepackage{verbatim}
\usepackage{tikz}
\usetikzlibrary{automata,positioning}
\usepackage{multicol}
\usepackage{graphicx}
\usepackage[colorlinks,urlcolor=blue]{hyperref}
\usepackage[stable]{footmisc}
\usepackage{ dsfont }

\usepackage{xparse}
\usepackage{ifthen}
\usepackage{bm}
\usepackage{color}

\usepackage{algorithm}
\usepackage{algpseudocode}

\DeclareMathOperator{\sign}{sign}
\DeclareMathOperator{\intt}{int}
\DeclareMathOperator{\conv}{conv}

\newtheorem{theorem}{Theorem}[section]
\newtheorem{lemma}{Lemma}

\begin{document}
\section{Description Of A Task}

Let's consider a following task:

$$\min_{(x,y)}\left\{f(x,y)|(x,y) \in Q\right\},$$
where $f$ is a convex function, $Q$ - is a square on the plane.

Let's consider a following method. One solves task of minimization for a function $g(x) = f\left(x, y_0 = \frac{a}{2}\right)$ on a segment $[0, a]$ with an accuracy $\delta$ on function. After that one calculates a sub-gradient in a received point and chooses the rectangle which the sub-gradient "does not look" in. Similar actions are repeated for a vertical segment. As a result we have the square decreased twice. Let's find a possible value of error $\delta_0$ for task on segment and a sufficient iteration's number $N$ to solve the initial task with accuracy $\epsilon$ on function.

Let's describe an algorithm formally. See pseudo-code \ref{alg:Example}.

\begin{algorithm}[H]
\caption{Algorithm of the method}\label{alg:Example}
\begin{algorithmic}[1]
\Function{Method}{convex function $f$, square $Q = [a, b]\times[c,d]$}

$x_0:= solve(g = f(\cdot, \frac{c+d}{2}), [a,b], \delta)$
 
$g = subgradient(f, (x_0, \frac{c+d}{2}))$
 
\If{g[1] > 0}
\State$Q := [a,b] \times [c, \frac{c+d}{2}]$ 
\Else
\State$Q := [a,b] \times [\frac{c+d}{2}, d]$ 
\EndIf

 $y_0:= solve(g = f(\frac{a+b}{2}, \cdot), [c,d], \delta)$
 
 $g := subgradient(f, (\frac{a+b}{2}), y_0)$
 
 \If{g[0] > 0}
 \State$Q := [a, \frac{a+b}{2}] \times [c, d]$ 
\Else
\State$Q := [\frac{a+b}{2},b] \times [c, d]$ 
\EndIf
\If{StopRec() == False}
\State Method($f$, $Q$) 
\EndIf
\Return $(\frac{a+b}{2}, \frac{c+d}{2})$
\EndFunction 
 \end{algorithmic}
\end{algorithm}

\section{Algorithm correctness}
Let's $\textbf{x}_0$ is solution of the task on segment, $Q_1$ is choosed rectangle, $Q_2$ is not choosed rectangle.

\subsection{Zero Error}

\begin{lemma}\label{l1}
If the optimization task on segment is solved with zero error and the $f$ is convex and differentiable at a point-solution, rectangle with solution of initial task was choosed correct.
\end{lemma}
\begin{proof}
From sub-gradient definition, $\textbf{x}^* \in \{x|(\textbf{g}(\textbf{x}_0), \textbf{x}_0 - \textbf{x}^*) \geq 0)\}$. Lemma's statement follows from it and a fact that the first (or the second for vertical segment) gradient's component in point $\textbf{x}_0$ is zero.
\end{proof}

The method does not work for all convex functions even for zero error on segment. Let's consider following example.

\textbf{Example 1}.
\begin{equation}
f(x,y) = \max\{x-2y, y-2x\}, \\
Q = [-1, 1]^2
\end{equation}

Function $f$ is convex as maximum of affine functions on $x$ and $y$. A solution of task on horizontal segment $[-1, 1]\times\{0\}$ is point $(0, 0)$. Its subdifferential is $$\partial f(0,0) = \conv\left\{(1, -2)^\top, (-2,1)^\top\right\}.$$ 

So if one takes a subgradient $(1,-2)^\top$ then a bottom rectangle will be choosed. But optimal value is point $(1,1)$ and there is not it in choosed rectangle. Therefore, this method cannot give a solution of initial task with error less than $\frac{1}{2}$.
\subsection{Nonzero Error}
\begin{theorem}
Let's the $f$ has continuous derivative on the square. Then there is a neighbourhood of a solution of optimization task on segment such as a choice of rectangle will not change if one use any point from the   neighbourhood.
\end{theorem}
\begin{proof}
Let's consider a case when we work with horizontal segment. Case with vertical segment is considered analogously. Then we are interesting in $f_y'(x_0, y_0)$. If $\textbf{x}_0$ is not solution of initial task, then $f_y'(x_0, y_0) \neq 0$. 

From a continuity of the derivative:

$$\lim\limits_{\delta \rightarrow 0}f_y'(x_0+\delta, y_0) = f_y'(x_0, y_0)$$

Therefore, 

$$\exists \delta_0:\forall \textbf{x}_\delta\in B_{\delta_0}(x_0)\times y_0\Rightarrow \sign(f_y'(\textbf{x}_\delta)) = \sign(f_y'(\textbf{x}_0))$$

From it and lemma 1 theorem's statement follows.

\end{proof}

\section{Error's Value}

Let's find possible value of error's value $\delta_0$. Rectangles are defined correctly for a horizontal optimization task, if:

\begin{equation}\label{1}
\forall\delta : |\delta| < \delta_0 \Rightarrow f'_y(\textbf{x}_0)f'_y(x_0+\delta, y_0) > 0
\end{equation}

Analogically, for a vertical segment:
\begin{equation}
\forall\delta: |\delta| < \delta_0 \Rightarrow f'_x(\textbf{x}_0)f'_x(x_0, y_0+\delta) > 0
\end{equation}
\begin{theorem}
Let's function $f$ is convex and differentiable and current rectangle is $[a,b]\times[c,d]$.

\textbf{For horizontal segment:}  There is $f''_{xy}$ on the segment. Rectangle is defined in point $(x_0 + \delta, y_0)$ correctly if one meet following condition:
\begin{equation}
\boxed{\delta_0 < \frac{|f'_y(\textbf{x}_0)|}{\max\limits_{t\in [a,b]}|f''_{xy}(t, y_0)|}}
\end{equation}

\textbf{For vertical segment:}  There is $f''_{yx}$ on the segment. Rectangle is defined in point $(x_0, y_0+\delta)$ correctly if one meet following condition:
\begin{equation}
\boxed{\delta_0 < \frac{|f'_x(\textbf{x}_0)|}{\max\limits_{t\in [c,d]}|f''_{yx}(x_0, t)|}}
\end{equation}

\end{theorem}
\begin{proof}

Let's prove this statement for horizontal segment.

Rewrite the condition (1) using Taylor formula:

$$\forall \delta:|\delta|\leq\delta_0\Rightarrow f'_y(\textbf{x}_0)\left(f'_y(\textbf{x}_0)+f''_{xy}\left(\textbf{x}_0+(\theta\delta,0)^\top\right)\delta\right) > 0,$$
where $\theta\in(0,1)$

Using the written above inequality we have a following inequality for $\delta_0$:

$$\delta_0<\frac{|f'_y(\textbf{x}_0)|}{\max\limits_{\theta \in [-1, 1]}|f''_{xy}(x_0 + \theta\delta_0, y_0)|}$$

It and an obvious inequality $\max\limits_{\theta \in [-1, 1]}|f''_{xy}(x_0 + \theta\delta_0, y_0)| < \max\limits_{t\in [a,b]}|f''_{xy}(t, y_0)|$ proves (3). Inequality (4) are proved similar.
\end{proof}

\begin{theorem}
Let's function $f$ is convex and has $L$-Lipschitz continuous gradient. 

And on horizontal segment $\exists M_1  : \forall \textbf{x}\Rightarrow|f_y'(\textbf{x})| > M_1$. Then rectangle is defined correctly if the possible value of error is not more $\frac{M_1}{L}$.

And on vertical segment $\exists M_2  : \forall \textbf{x}\Rightarrow|f_x'(\textbf{x})| > M_2$. Then rectangle is defined correctly if the possible value of error is not more $\frac{M_2}{L}$
\end{theorem}
\begin{proof}
Condition (1) is met if there is a derivative $f'_y(x_0+\delta, y_0)$ in a neighbourhood of 
$f'_y(\textbf{x}_0)$ with radius $\left|f'_y(\textbf{x}_0)\right|$:

$$\left|f'_y(\textbf{x}_0) - f'_y(x_0+\delta, y_0)\right|<\left|f'_y(\textbf{x}_0)\right|$$

The $L$-Lipschitz continuity gives following inequality:

$$\left|f'_y(\textbf{x}_0) - f'_y(x_0+\delta, y_0)\right| \leq L|\delta|$$

Theorem's estimate for vertical segment follows from two written above inequality. Inequality for vertical segment is proved similarly.
\end{proof}

\section{Number of iterations}

Following estimates are correct if each iterations was correct (a rectangle is selected correctly on each iterations).

\begin{theorem}
If function $f$ is convex and $L$-Lipschitz continuous, then for to solve initial task with accuracy $\epsilon$ on function one has to make following iteration's number:
$$N = \left\lceil\log_2\frac{La}{\sqrt{2}\epsilon}\right\rceil,$$
where $a$ is a size of the initial square $Q$
\end{theorem}
\begin{proof}
$$|f(\textbf{x}^*) - f(\textbf{x})| < L|\textbf{x}^* - \textbf{x}|$$

After $N$ iterations we have a square with size $\frac{a}{2^N}$. That's why if we choose a squares center as proximal solution we have following estimates:
$$|\textbf{x}^* - \textbf{x}| \leq \frac{a}{\sqrt{2}}2^{-N}$$
$$|f(\textbf{x}^*) - f(\textbf{x})| < La2^{-N}$$

Therefore, for accuracy epsilon following number of iterations is sufficient:
$$N > \log_2\frac{La}{\sqrt{2}\epsilon}$$

\end{proof}

But any convex function is locally Lipschitz continuous at all $x \in \intt Q$. Therefore, we have following theorem.

\begin{theorem}
If function $f$ is convex and a solution $x^*\in \intt Q$, then for to solve initial task with accuracy $\epsilon$ on function one has to make following iteration's number:
$$N = \left\lceil\log_2\max\left\{\frac{a}{\epsilon_0(x^*)},\log\frac{La}{\sqrt{2}\epsilon}\right\}\right\rceil,$$
where $a$ is a size of the initial square $Q$, $\epsilon_0(x^*)$ is size of neighbourhood of $x^*$ which $f$ is $L$-Lipschitz continuous in.
\end{theorem}

\end{document}
