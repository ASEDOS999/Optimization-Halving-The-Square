 \documentclass[12pt]{article}
\usepackage[T2A]{fontenc}
\usepackage[utf8]{inputenc}       

\usepackage[english]{babel}
\usepackage{amsmath,amsfonts,amsthm,amssymb,amsbsy,amstext,amscd,amsxtra,multicol}
\usepackage{verbatim}
\usepackage{tikz}
\usetikzlibrary{automata,positioning}
\usepackage{multicol}
\usepackage{graphicx}
\usepackage[colorlinks,urlcolor=blue]{hyperref}
\usepackage[stable]{footmisc}
\usepackage{ dsfont }

\usepackage{xparse}
\usepackage{ifthen}
\usepackage{bm}
\usepackage{color}

\usepackage{algorithm}
\usepackage{algpseudocode}

\DeclareMathOperator{\sign}{sign}
\DeclareMathOperator{\intt}{int}
\DeclareMathOperator{\conv}{conv}

\newtheorem{theorem}{Theorem}[section]
\newtheorem{lemma}{Lemma}

\begin{document}
\section{Tests functions}

\subsection{Sinuses}
Functions $-\sin\frac{\pi x}{a}$ and $-\sin\frac{\pi x}{b}$ are convex on square $Q = [0,1]^2$ when $a,b\geq 1$. Therefore, a function $-A\sin\frac{\pi x}{a} - B\sin\frac{\pi y}{b}$ is convex for all $A,B\geq 0$ as cone combination of convex function.

Functions $x^n$ are convex and monotonously non-decreasing on $[0, 1]$ for all $n \in \mathbb{N}$ that's why functions $\left(-A\sin\frac{\pi x}{a} - B\sin\frac{\pi y}{b} + A + B + D\right)^n$ are convex for all $D\geq 0$.

Therefore, following function is convex:

$$f(x,y) = -A_1\sin\frac{\pi x}{a_1} - B_1\sin\frac{\pi x}{b_1} + \sum\limits_{n=2}^N\left(-A_n\sin\frac{\pi x}{a_n} - B_n\sin\frac{\pi y}{b_n} + A_n + B_n + D_n\right)^n,$$

where $A_i, B_i. D_i\geq 0$ and $a_i, b_i \geq 1$ for all $i = \overline{1, n}$.

The function $f$ is differentiable infinite times and we can use it to test the method.

Let's take $a_1 = \dots = a_n = a$ and $b_1 = \dots = b_n = b$:

$$f(x,y) = -A_1\sin\frac{\pi x}{a} - B_1\sin\frac{\pi x}{b} + \sum\limits_{n=2}^N\left(-A_n\sin\frac{\pi x}{a} - B_n\sin\frac{\pi y}{b} + A_n + B_n + D_n\right)^n,$$

where $A_i, B_i. D_i\geq 0$ for all $i = \overline{1, n}$ and $a, b \geq 1$.

Then functions derivative is:

$$f'_x(x,y) = \left(-A_1 - \sum\limits_{n=2}^NnA_n\left(-A_n\sin\frac{\pi x}{a} - B_n\sin\frac{\pi y}{b} + A_n + B_n + D_n\right)^{n-1}\right)\cdot$$
$$\cdot\frac{\pi}{a}\cos \frac{\pi x}{a}$$

$$f'_y(x,y) = \frac{\pi}{b}\left(-B_1 - \sum\limits_{n=2}^NnB_n\left(-A_n\sin\frac{\pi x}{a} - B_n\sin\frac{\pi y}{b} + A_n + B_n + D_n\right)^{n-1}\right)\cdot$$
$$\cdot\frac{\pi}{b}\cos \frac{\pi y}{b}$$

$$f''_{xy}(x,y) = \left(\sum\limits_{n=2}^Nn(n-1)A_nB_n\left(-A_n\sin\frac{\pi x}{a} - B_n\sin\frac{\pi y}{b} + A_n + B_n + D_n\right)^{n-2}\right)\cdot$$
$$\cdot\frac{\pi^2}{ab}\cos \frac{\pi x}{a}\cos \frac{\pi y}{b}$$

Using written above expressions we can give estimates for derivatives:

$$|f'_x|\Big|_{x = x_0} \geq \left(A_1 + \sum\limits_{n=2}^N n A_n D_n^{n-1}\right)\frac{\pi}{a}\Big|\cos \frac{\pi x_0}{a}\Big|$$
$$|f'_x|\Big|_{y = y_0} \geq \left(B_1 + \sum\limits_{n=2}^N n B_n D_n^{n-1}\right)\frac{\pi}{b}\Big|\cos \frac{\pi y_0}{b}\Big|$$
$$|f''_{xy}| \leq \frac{\pi^2}{ab}\left(\sum\limits_{n=2}^Nn(n-1)A_nB_n\left(A_n + B_n + D_n\right)^{n-2}\right)$$

Also we know solution of this task:

\begin{equation}
x^* = \begin{cases}
1, \text{if $a \geq 2$},\\
\frac{a}{2}, \text{else}
\end{cases}
\end{equation}

\begin{equation}
y^* = \begin{cases}
1, \text{if $b \geq 2$},\\
\frac{b}{2}, \text{else}
\end{cases}
\end{equation}
and task's value:
$$f^* = f(x^*, y^*).$$

We will use $a$ and $b$ from $[1, 2]$ and $N = 2$. That's why we can find task's value easy:

$$f^* = -A_1 - B_1 + \sum\limits_{n=2}^ND_n^n$$

Also we will use $N = 2$.

\subsection{Lipschitz continuous gradient}

Let's consider following function:

\begin{equation}
f(x) = x^2\cdot\begin{cases}
\frac{3}{2}, \text{if $x < 0$},\\
1, \text{if $x \geq 0$}
\end{cases}
\end{equation}

Function $f(x)$ is convex and has a Lipschitz continuous derivative but it is not twice differentiable at zero:

\begin{equation}
f'(x) = x\cdot\begin{cases}
3, \text{if $x < 0$},\\
2, \text{if $x \geq 0$}
\end{cases}
\end{equation}

Let's find $L$ for derivative:

\begin{equation}
\frac{|f'(x_1) - f'(x_2)|}{|x_1-x_2|} = \begin{cases}
3, \text{if $x_1, x_2 < 0$},\\
2, \text{if $x_1, x_2 \geq 0$}
\end{cases}\leq 3
\end{equation}
Let $x_1<0, x_2\geq 0$: 

$$\frac{|f'(x_1) - f'(x_2)|}{|x_1-x_2|} = \frac{|3x_1 - 2x_2|}{|x_1-x_2|} \leq \frac{|x_1|}{|x_1-x_2|} + 2\leq 3$$

As a result, we have:
$$L = 3$$

Let's consider following function:

$$g(x,y) = af(x)+bf(y) + \phi(x,y),$$
where $a,b\geq 0$ - constants, $\phi$ is a convex function with a Lipschitz continuous derivative. Therefore, $g$ has a Lipschitz continuous derivative with a Lipschitz constant $3(a+b) + L_\phi$.

Let's consider following functions:

$$g(x,y) = af(x) + bf(y) + (Ax+By)^2,$$

where $A, B$ and $a, b\geq 0$ are constants.

Then $g$ is convex and has a Lipschitz continuous derivative with a constant $3(a+b) + 2(|A_1|+|B_1|)^2 + |A_2| + |B_2|$ but it is not twice differentiable at zero.

$$g_x'(x,y) = \psi(x) x + 2A_1(A_1x+B_1y),$$
$$g_y'(x,y) = \psi(y) y + 2B_1(A_1x+B_1y),$$
where 
\begin{equation}
\psi(x) = \begin{cases}
3, \text{if $x < 0$},\\
2, \text{if $x \geq 0$}
\end{cases}
\end{equation}

Then a point (0,0) is optimal. Also it is only optimal point. 

Let's consider minimization task for $g$ on square $Q = [x_1, x_2]\times [y_1, y_2]$ such as $0\in Q$. Estimates for derivatives:

\begin{eqnarray}
|g_x'(x,y)|\Big|_{x=x_0} = |(\psi(x_0) + 2A_1^2)x_0 + 2A_1B_1y| \geq \nonumber\\
\geq \begin{cases}
0, \text{if $-\frac{(\psi(x_0) + A_1^2)x_0}{2A_1B_1}\in [y_1, y_2]$ },\\
|(\psi(x_0) + 2A_1^2)x_0 + 2A_1B_1y_2|, \text{if $|(\psi(x_0) + 2A_1^2)x_0 + 2A_1B_1y|<0$},\\ 
|(\psi(x_0) + 2A_1^2)x_0 + 2A_1B_1y_1|, \text{if $|(\psi(x_0) + 2A_1^2)x_0 + 2A_1B_1y|>0$},\\
\end{cases}
\end{eqnarray}

For $g_y'$ we have similar estimate:

\begin{eqnarray}
|g_y'(x,y)|\Big|_{y=y_0} \geq\nonumber\\ 
\geq \begin{cases}
0, \text{if $-\frac{(\psi(y_0) + 2B_1^2)y_0}{2A_1B_1}\in [x_1, x_2]$ },\\
|(\psi(y_0) + 2B_1^2)y_0 + 2A_1B_1x_2|, \text{if $|(\psi(y_0) + 2B_1^2)y_0 + 2A_1B_1x|<0$},\\ 
|(\psi(y_0) + 2B_1^2)y_0 + 2A_1B_1x_1|, \text{if $|(\psi(y_0) + 2B_1^2)y_0 + 2A_1B_1x|>0$},\\
\end{cases}
\end{eqnarray}
\end{document}
